\section{Introduction}

This thesis investigates the efficacy of ranking scalar adjectives from corpus data using monolingual corpus and data derived from bilingual sources. We make three major contributions:

\begin{enumerate}
	\item successfully reproduce the state of the art in adjective ranking using monolingual corpus
	\item offer two alternative formulations of ranking adjectives using monolingual corpus, each one achieves parity with the state of the art. 
	\item successfully incorporate bilingual data with monolingual data, and outperforms the state of the art by a non-trivial amount.
\end{enumerate}

Additionally, all code and data needed to reproduce the successful experiments in this thesis are distributed and can be found at: https://github.com/lingxiao/adj-relation. 

This thesis is organized as follows: in the rest of this chapter we will motivate the problem of adjective ranking and give an overview of prior work. In chapter 2 we will discuss the two sources of data we consider in this thesis, as well as how the gold standards are procured. We will also discuss how the rankings are typically evaluated. Chapter 3 gives thorough review of the state of the art method, and implementation details that arose while we reproduced the method. In chapter 4 we offer a very different formulation of the problem at a high level. Chapter 5 will present a simple baseline in the spirit of this reformulation that performs surprisingly well. Chapter 6 refines the baseline using a variety of models, and finally in chapter 7 we combine the models and outperforms the state of the art by a significant amount. This chapter will conclude with several motivations for future work. Chapters 8 and 9 are optional reading for those who wish to consider other points of view on the problem. Chapters 8 will gives a formulation that is on parity with the previous state of the art, although this model uses monolingual data only. Chapter 9 gives a detailed account of a failed attempt at ranking adjectives, readers should note that this chapter is meant to amuse rather than to inform.  

Now we will set up the problem of ranking adjectives. Linguistic scale is a set of words of the same grammatical category that can be ordered by their expressive strength or degree of informativeness \cite{sheinman2009adjscales}. Ranking adjectives over such a scale is a particularly important task in sentiment analysis, recognizing textual entailment, question answering, summarization, and automatic text understanding and generation. For instance, understanding the word ``great" is a stronger indicator of quality than the word ``good" could help refine the decision boundary between four star reviews versus five star one. However, current lexical resources such as WordNet do not provided such crucial information about the intensity order of adjectives.

Past work approached this problem in two ways: distributional and linguistic-pattern based. \newcite{kim2013deriving} showed that word vectors learned by a recurrent neural network language model can determine scalar relationships among adjectives. Specifically, given a line connecting a pair of antonyms, they posited that intermediate adjective word vectors extracted along this line should correspond to some intensity scale determined by the antonyms. The quality of the extracted relationship is evaluated using indirect yes/no question answer pairs, and they achieved 72.8\% pairwise accuracy over 125 pairs.

While distributional methods infer pairwise relationship between adjectives based on how they occur in the corpus separately, linguistic-pattern based approaches decides this relationship using their joint co-occurence around pre-determined patterns \cite{sheinman2009adjscales,schulam2010automatically,sheinman2012refining} . For example, the phrase ``good but not great" suggests good is less intense than great. These patterns are hand-curated for their precision and unsurprisingly enjoy high accuracy. However, they suffer from low recall because the amount of data needed to relate a pair of adjectives is exponential in length of the pattern, while such patterns are no less than four to five words long. 

\newcite{demelo:13} addressed this data sparsity problem by exploiting the transitive property of partial orderings to determine unobserved pairwise relationships. They observed that in order to deduce an ordering over good, great, and excellent, it suffices to observe good is less than great, and great is less than excellent. Then by transitive property of the ordering we conclude good is also less than excellent. This fixed relationship among adjectives is enforced by a mixed integer linear program (MILP). Banal and de Melo tested their approach on 91 adjective clusters, where the average number of adjectives in each cluster is just over three, and each cluster is ranked by a set of annotators. They reported 69.6\% pair-wise accuracy and 0.57 average Kendall's tau. Now we will give a detailed explanation of how the monolingual and bilingual corpus is constructed, how the gold standards are curated, and how performance is evaluated.\newpage
